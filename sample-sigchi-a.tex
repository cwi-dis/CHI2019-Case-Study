\documentclass[sigchi-a, authorversion]{acmart}
\usepackage{booktabs} % For formal tables
\usepackage{ccicons}  % For Creative Commons citation icons
\usepackage{todonotes}

% Copyright
%\setcopyright{none}
%\setcopyright{acmcopyright}
\setcopyright{acmlicensed}
%\setcopyright{rightsretained}
%\setcopyright{usgov}
%\setcopyright{usgovmixed}
%\setcopyright{cagov}
%\setcopyright{cagovmixed}


% DOI
\acmDOI{10.475/123_4}

% ISBN
\acmISBN{123-4567-24-567/08/06}

%Conference
\acmConference[CHI'19]{ACM CHI Conference}{May 2019}{Glasgow, Scotland, United Kingdom}
\acmYear{2019}
\copyrightyear{2019}

\acmPrice{15.00}

%\acmBadgeL[http://ctuning.org/ae/ppopp2016.html]{ae-logo}
%\acmBadgeR[http://ctuning.org/ae/ppopp2016.html]{ae-logo}

\begin{document}
\title{From the Lab to the OB Truck: Object-Based Broadcasting at the FA Cup in Wembley Stadium}

\author{First Author}
\affiliation{%
  \institution{University of Author}
  \city{Authortown}
  \state{CA}
  \postcode{94022}
  \country{USA} }
\email{author1@anotherco.edu}

\author{Second Author}
\affiliation{%
  \position{VP, Authoring}
  \institution{Authorship Holdings, Ltd.}
  \city{Awdur}
  \postcode{SA22 8PP}
  \country{UK}}
\email{author2@author.ac.uk}

\author{Third Author \\
  Fourth Author}
\affiliation{%
  \institution{L\={e}khaka Labs}
  \city{Bengaluru} \postcode{560 080} \country{India}}
\email{author3@another.com}
\email{author4@another.com}

\author{Fifth Author}
\affiliation{\institution{YetAuthorCo, Inc.}
  \city{Authortown} \state{BC}
  \postcode{V6M 22P} \country{Canada}}
\email{author5@anotherco.com}

\author{Sixth Author}
\affiliation{\institution{Universit\'e de Auteur-Sud}
  \city{Auteur} \postcode{40222} \country{France}}
\email{author6@author.fr}

\author{Seventh Author}
\affiliation{\institution{University of Umbhali}
  \city{Pretoria} \country{South Africa}}
\email{author7@umbhaliu.ac.za}

% The default list of authors is too long for headers.
\renewcommand{\shortauthors}{F. Author et al.}


%
% The code below should be generated by the tool at
% http://dl.acm.org/ccs.cfm
% Please copy and paste the code instead of the example below.
%
\begin{CCSXML}
<ccs2012>
 <concept>
  <concept_id>10010520.10010553.10010562</concept_id>
  <concept_desc>Computer systems organization~Embedded systems</concept_desc>
  <concept_significance>500</concept_significance>
 </concept>
 <concept>
  <concept_id>10010520.10010575.10010755</concept_id>
  <concept_desc>Computer systems organization~Redundancy</concept_desc>
  <concept_significance>300</concept_significance>
 </concept>
 <concept>
  <concept_id>10010520.10010553.10010554</concept_id>
  <concept_desc>Computer systems organization~Robotics</concept_desc>
  <concept_significance>100</concept_significance>
 </concept>
 <concept>
  <concept_id>10003033.10003083.10003095</concept_id>
  <concept_desc>Networks~Network reliability</concept_desc>
  <concept_significance>100</concept_significance>
 </concept>
</ccs2012>
\end{CCSXML}

\ccsdesc[500]{Computer systems organization~Embedded systems}
\ccsdesc[300]{Computer systems organization~Redundancy}
\ccsdesc{Computer systems organization~Robotics}
\ccsdesc[100]{Networks~Network reliability}


\begin{abstract}
    \ldots
\end{abstract}


\keywords{Authors' choice; of terms; separated; by
  semicolons; include commas, within terms only; required.}



\maketitle

\section{Introduction}
 \todo[inline]{Tom to provide an introduction of the paper and how it will tell the journey from the lab to the truck}

 The world of live-broadcasting is a world of well-defined workflows comprised
 of relatively few but time-critical tasks. The roles inside an
 Outside-Broadcast (OB) truck at a live sporting event are also very
 well-defined and subdivided into a few discrete stations where each station
 has to rely on other stations to complete their tasks in a timely manner as to
 not incur any delay, which would start to propagate and accumulate down the
 line and potentially cause transmission delays.

 This case study presents the idea behind, the iterative development and the
 final deployment at a live sporting event of a new, web-based platform for
 injecting customisable, pre-defined objects into a running livestream. More
 specifically, we detail how the initial idea came about, how the platform
 evolved over several iterations by interviewing professionals from the world
 of broadcasting from a rough prototype to a multi-user platform with hardware
 integration, optimised for collaboration and efficiency. We also present a
 series of trials and experiments that very completed along the way,
 culminating in a final deployment at the FA Cup final at Wembley stadium in
 London in Spring 201. The paper concludes with a discussion section reflecting
 on the process and presenting opporunities for future development.

\section{Related Work}
 \todo[inline]{Jie to write the related work section}

An Outside Broadcasting (OB) truck is the most common unit for live sports broadcasting. With its mobility, OB trucks can access any location and work as a drive-in temporary production control center during live broadcasting with complete video and audio facilities. An OB truck typically has a wall of monitors shared by all the staff on the truck, video production switcher controlled by the director, audio mixer, a team in charge of recording and playback decks, and a team responsible for live graphics and so on. It enables the production team to bring the audience an authentic visual report of an event as it is happening \cite{owens2012, owens2015}.

OB trucks vary in sizes depending on the scale of coverage and the nature of the event. For a sport event, the coverage includes dozens of stationary cameras, a couple of handheld cameras, cameras on motorcycles to capture the main athletes and one or two helicopters with cameras to shoot the "long-shot" of the scene \cite{owens2012, li2018}. Today, OB trucks work perfectly to optimize the live content on TV screens. The production team on OB trucks use the minutes as reference points throughout the live broadcasting process following a pre-scripted "running order document", which defines in detail where graphics, visual sources and sound come from and when they should be "on-air". During the live broadcasting, especially for an intense sports event, the staff on the OB truck were utterly focusing on their own tasks. The workload of the producer, the director and the replay operators was extremely heavy to make everything work for TV screens.
 \cite{li2018}. 
 
Today, another challenge the TV industry facing is not only about optimizing the content on TV screens, but to tailor the content so that it works for other companion screens (e.g., smartphones and tablets) as though it had been uniquely created for them. Content on companion screens are customizable to access additional information next to the TV screen, enabling viewers to have interactive and immersive TV viewing experiences\cite{bentley2017, dowell2015}. However, the workload of current live broadcasting makes it difficult to produce extra content for companion screens. As companion screens continue to be integrated into standard television, new technologies are required to produce content for multiscreens, for instance, object-based broadcasting\cite{li2018, armstrong2014}.

Object-based broadcasting (OBB) approach allows the content of a TV program to adapt to the requirements of different viewers on multiscreens, without separately producing many different versions of the program. The "object", here, refers to different media units that are used to make a TV program \cite{armstrong2014}. The OBB approach involves breaking down a program into separate content objects, typically including graphics, audio, video, background music, dialogues, subtitles, sound/visual effects, etc., and including a metadata to describe how these objects can be assembled on multiscreens. It enables the production of a flexible, personalized, and responsive program without increasing the production workload\cite{kegel2017, williams2016}.

Some recent studies have explored object-based production in different use cases. Cox et al. \cite{cox2017} developed a personalized interactive cooking application called \textit{CAKE}. \textit{CAKE} supports an interactive dialogue between a viewer and a cooking show. \textit{CAKE} can integrate multiple recipes selected by the viewer and automatically generate a step-by-step cooking plan. In this way, the show is adjustable according to the viewer's pace. \textit{Squeezebox} developed by BBC Research \& Development \cite{BBC2015} is an object-based tool that enables rapid re-editing the duration of the content. \textit{Squeezebox} can automatically analyze and segment the footage into individual shots. The production team can mark the priority of each shot, determining whether the footage will be cut or preserved as the duration is reduced. Puentes et al. \cite{puentes2017} developed a flexible visual authoring tool. By dragging and dropping static or dynamic "components" (e.g., texts, images, videos, audio, fonts, etc.) into "containers" (e.g., defined regions on the screens), developers and designers become time-and-cost-efficient in creating interactive TV applications. Most recently, Smith et al. \cite{smith2018} presented a cross-platform object-based application that provides a graphical environment for the authoring of video narratives that can be dynamically sequenced and composited at viewing time, based upon the interactions, or context, of the audience.

\section{Preparation}
 \todo[inline]{Pablo to bring a summary from the deliverable}

\section{Experiment}
 \todo[inline]{Pablo to bring a summary from the deliverable}


\section{Discussion}
 \todo[inline]{Tom and Jie to identify a number of topics for discussion; see D4.6}

\section{Conclusion}

\ldots

\bibliography{sample-bibliography-sigchi-a}
\bibliographystyle{ACM-Reference-Format}

\end{document}
